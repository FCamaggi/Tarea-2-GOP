\usepackage[spanish]{babel}
\usepackage[utf8]{inputenc}
\usepackage[T1]{fontenc}
\usepackage{amsmath, amssymb, amsthm}
\usepackage{graphicx}
\usepackage{hyperref}
\usepackage{xcolor}
\usepackage{enumitem}
\usepackage{booktabs}
\usepackage{array}
\usepackage{longtable}
\usepackage{multirow}
\usepackage{caption}
\usepackage{subcaption}
\usepackage{geometry}
\usepackage{setspace}
\usepackage{float}
\usepackage{pdflscape}
\usepackage{csvsimple}
\usepackage{pgfplotstable}
\usepackage{siunitx} % Para formato de números

\geometry{a4paper, margin=2.5cm}

% Configuraciones adicionales
\setlength{\parindent}{0pt}
\setlength{\parskip}{1em}
\renewcommand{\baselinestretch}{1.5}

% Comandos personalizados
\newcommand{\question}[1]{\section*{Pregunta #1}}
\newcommand{\subquestion}[1]{\subsection*{Subpregunta #1}}
\newcommand{\answer}{\noindent\textbf{Respuesta:}\par}

% Comando para importar tablas CSV con formato automático
\newcommand{\importCSV}[4]{%
    % #1: ruta archivo CSV, #2: número de columnas, #3: cabecera de tabla, #4: comandos por fila
    \csvreader[
        tabular=#2,
        table head=\toprule #3 \\\midrule,
        command=#4,
        late after line=\\,
        table foot=\bottomrule,
        respect dollar=false,
        respect percent=false
    ]{#1}{}{} % Los {} vacíos son para el mapping de columnas y el contenido
}

% Comando para controlar automáticamente el tamaño de tablas grandes
\newcommand{\adjustableTable}[3]{%
    % #1: ancho de tabla (ej: 1.0\textwidth), #2: caption, #3: contenido
    \begin{table}[H]
        \centering
        \caption{#2}
        \resizebox{#1}{!}{#3}
    \end{table}
}